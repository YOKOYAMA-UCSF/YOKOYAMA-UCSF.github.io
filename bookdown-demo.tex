% Options for packages loaded elsewhere
\PassOptionsToPackage{unicode}{hyperref}
\PassOptionsToPackage{hyphens}{url}
%
\documentclass[
]{book}
\usepackage{amsmath,amssymb}
\usepackage{iftex}
\ifPDFTeX
  \usepackage[T1]{fontenc}
  \usepackage[utf8]{inputenc}
  \usepackage{textcomp} % provide euro and other symbols
\else % if luatex or xetex
  \usepackage{unicode-math} % this also loads fontspec
  \defaultfontfeatures{Scale=MatchLowercase}
  \defaultfontfeatures[\rmfamily]{Ligatures=TeX,Scale=1}
\fi
\usepackage{lmodern}
\ifPDFTeX\else
  % xetex/luatex font selection
\fi
% Use upquote if available, for straight quotes in verbatim environments
\IfFileExists{upquote.sty}{\usepackage{upquote}}{}
\IfFileExists{microtype.sty}{% use microtype if available
  \usepackage[]{microtype}
  \UseMicrotypeSet[protrusion]{basicmath} % disable protrusion for tt fonts
}{}
\makeatletter
\@ifundefined{KOMAClassName}{% if non-KOMA class
  \IfFileExists{parskip.sty}{%
    \usepackage{parskip}
  }{% else
    \setlength{\parindent}{0pt}
    \setlength{\parskip}{6pt plus 2pt minus 1pt}}
}{% if KOMA class
  \KOMAoptions{parskip=half}}
\makeatother
\usepackage{xcolor}
\usepackage{color}
\usepackage{fancyvrb}
\newcommand{\VerbBar}{|}
\newcommand{\VERB}{\Verb[commandchars=\\\{\}]}
\DefineVerbatimEnvironment{Highlighting}{Verbatim}{commandchars=\\\{\}}
% Add ',fontsize=\small' for more characters per line
\usepackage{framed}
\definecolor{shadecolor}{RGB}{248,248,248}
\newenvironment{Shaded}{\begin{snugshade}}{\end{snugshade}}
\newcommand{\AlertTok}[1]{\textcolor[rgb]{0.94,0.16,0.16}{#1}}
\newcommand{\AnnotationTok}[1]{\textcolor[rgb]{0.56,0.35,0.01}{\textbf{\textit{#1}}}}
\newcommand{\AttributeTok}[1]{\textcolor[rgb]{0.13,0.29,0.53}{#1}}
\newcommand{\BaseNTok}[1]{\textcolor[rgb]{0.00,0.00,0.81}{#1}}
\newcommand{\BuiltInTok}[1]{#1}
\newcommand{\CharTok}[1]{\textcolor[rgb]{0.31,0.60,0.02}{#1}}
\newcommand{\CommentTok}[1]{\textcolor[rgb]{0.56,0.35,0.01}{\textit{#1}}}
\newcommand{\CommentVarTok}[1]{\textcolor[rgb]{0.56,0.35,0.01}{\textbf{\textit{#1}}}}
\newcommand{\ConstantTok}[1]{\textcolor[rgb]{0.56,0.35,0.01}{#1}}
\newcommand{\ControlFlowTok}[1]{\textcolor[rgb]{0.13,0.29,0.53}{\textbf{#1}}}
\newcommand{\DataTypeTok}[1]{\textcolor[rgb]{0.13,0.29,0.53}{#1}}
\newcommand{\DecValTok}[1]{\textcolor[rgb]{0.00,0.00,0.81}{#1}}
\newcommand{\DocumentationTok}[1]{\textcolor[rgb]{0.56,0.35,0.01}{\textbf{\textit{#1}}}}
\newcommand{\ErrorTok}[1]{\textcolor[rgb]{0.64,0.00,0.00}{\textbf{#1}}}
\newcommand{\ExtensionTok}[1]{#1}
\newcommand{\FloatTok}[1]{\textcolor[rgb]{0.00,0.00,0.81}{#1}}
\newcommand{\FunctionTok}[1]{\textcolor[rgb]{0.13,0.29,0.53}{\textbf{#1}}}
\newcommand{\ImportTok}[1]{#1}
\newcommand{\InformationTok}[1]{\textcolor[rgb]{0.56,0.35,0.01}{\textbf{\textit{#1}}}}
\newcommand{\KeywordTok}[1]{\textcolor[rgb]{0.13,0.29,0.53}{\textbf{#1}}}
\newcommand{\NormalTok}[1]{#1}
\newcommand{\OperatorTok}[1]{\textcolor[rgb]{0.81,0.36,0.00}{\textbf{#1}}}
\newcommand{\OtherTok}[1]{\textcolor[rgb]{0.56,0.35,0.01}{#1}}
\newcommand{\PreprocessorTok}[1]{\textcolor[rgb]{0.56,0.35,0.01}{\textit{#1}}}
\newcommand{\RegionMarkerTok}[1]{#1}
\newcommand{\SpecialCharTok}[1]{\textcolor[rgb]{0.81,0.36,0.00}{\textbf{#1}}}
\newcommand{\SpecialStringTok}[1]{\textcolor[rgb]{0.31,0.60,0.02}{#1}}
\newcommand{\StringTok}[1]{\textcolor[rgb]{0.31,0.60,0.02}{#1}}
\newcommand{\VariableTok}[1]{\textcolor[rgb]{0.00,0.00,0.00}{#1}}
\newcommand{\VerbatimStringTok}[1]{\textcolor[rgb]{0.31,0.60,0.02}{#1}}
\newcommand{\WarningTok}[1]{\textcolor[rgb]{0.56,0.35,0.01}{\textbf{\textit{#1}}}}
\usepackage{longtable,booktabs,array}
\usepackage{calc} % for calculating minipage widths
% Correct order of tables after \paragraph or \subparagraph
\usepackage{etoolbox}
\makeatletter
\patchcmd\longtable{\par}{\if@noskipsec\mbox{}\fi\par}{}{}
\makeatother
% Allow footnotes in longtable head/foot
\IfFileExists{footnotehyper.sty}{\usepackage{footnotehyper}}{\usepackage{footnote}}
\makesavenoteenv{longtable}
\usepackage{graphicx}
\makeatletter
\def\maxwidth{\ifdim\Gin@nat@width>\linewidth\linewidth\else\Gin@nat@width\fi}
\def\maxheight{\ifdim\Gin@nat@height>\textheight\textheight\else\Gin@nat@height\fi}
\makeatother
% Scale images if necessary, so that they will not overflow the page
% margins by default, and it is still possible to overwrite the defaults
% using explicit options in \includegraphics[width, height, ...]{}
\setkeys{Gin}{width=\maxwidth,height=\maxheight,keepaspectratio}
% Set default figure placement to htbp
\makeatletter
\def\fps@figure{htbp}
\makeatother
\setlength{\emergencystretch}{3em} % prevent overfull lines
\providecommand{\tightlist}{%
  \setlength{\itemsep}{0pt}\setlength{\parskip}{0pt}}
\setcounter{secnumdepth}{5}
\usepackage{booktabs}
\usepackage{amsthm}
\makeatletter
\def\thm@space@setup{%
  \thm@preskip=8pt plus 2pt minus 4pt
  \thm@postskip=\thm@preskip
}
\makeatother
\ifLuaTeX
  \usepackage{selnolig}  % disable illegal ligatures
\fi
\usepackage[]{natbib}
\bibliographystyle{apalike}
\IfFileExists{bookmark.sty}{\usepackage{bookmark}}{\usepackage{hyperref}}
\IfFileExists{xurl.sty}{\usepackage{xurl}}{} % add URL line breaks if available
\urlstyle{same}
\hypersetup{
  pdftitle={A Minimal Book Example},
  pdfauthor={Yihui Xie},
  hidelinks,
  pdfcreator={LaTeX via pandoc}}

\title{A Minimal Book Example}
\author{Yihui Xie}
\date{2024-03-15}

\begin{document}
\maketitle

{
\setcounter{tocdepth}{1}
\tableofcontents
}
\hypertarget{prerequisites}{%
\chapter{Prerequisites}\label{prerequisites}}

This is a \emph{sample} book written in \textbf{Markdown}. You can use anything that Pandoc's Markdown supports, e.g., a math equation \(a^2 + b^2 = c^2\).

The \textbf{bookdown} package can be installed from CRAN or Github:

\begin{Shaded}
\begin{Highlighting}[]
\FunctionTok{install.packages}\NormalTok{(}\StringTok{"bookdown"}\NormalTok{)}
\CommentTok{\# or the development version}
\CommentTok{\# devtools::install\_github("rstudio/bookdown")}
\end{Highlighting}
\end{Shaded}

Remember each Rmd file contains one and only one chapter, and a chapter is defined by the first-level heading \texttt{\#}.

To compile this example to PDF, you need XeLaTeX. You are recommended to install TinyTeX (which includes XeLaTeX): \url{https://yihui.name/tinytex/}.

\hypertarget{the-lab}{%
\chapter{The Yokoyama Lab}\label{the-lab}}

Yokoyama Lab Overview

\hypertarget{lab-mission}{%
\section{Yokoyama Lab Mission}\label{lab-mission}}

\hypertarget{lab-philosophy}{%
\section{Our Philosophy}\label{lab-philosophy}}

\hypertarget{our-research}{%
\section{Our Research}\label{our-research}}

\hypertarget{lab-values}{%
\section{Our Core Values}\label{lab-values}}

The way we conduct ourselves is a critical part of how we conduct science. Lab members will strive to uphold the following values:

\begin{itemize}
\item
  \textbf{Integrity}: Like many things in the world, science is built on trust and honesty. We will pursue rigorous and transparent research, ensuring our work is credible and reliable.
\item
  \textbf{Humility}: By accepting that we may not decidedly know or understand everything, we are open to new perspectives. We are willing to admit we made mistakes and share what we have learned from them.
\item
  \textbf{Kindness}: Science involves skepticism and criticism, but that does not mean it cannot be done with care and empathy. We hope to be respectful members of the scientific community, being considerate and supportive to our colleagues at all levels.
\item
  \textbf{Diversity}: Science should reflect the vast perspectives, experiences and people that exist in the world. Our lab hopes to be inclusive to all, and promote equity throughout science.
\end{itemize}

\hypertarget{our-location}{%
\section{Our Location}\label{our-location}}

\hypertarget{lab-culture}{%
\chapter{Lab Culture}\label{lab-culture}}

This chapter describes the structure and culture within the PALM Lab. The goal is to be clear about expectations for the members of the lab (ones that I have, as well as for each other) and importantly for myself.

\begin{quote}
\emph{This section aims to provide a picture of what mentoring and working with me will be like in the PALM Lab. It will continually be in-progress, shaped by (and including) feedback from lab members and others alike. -- William}
\end{quote}

\hypertarget{lab-environment}{%
\section{Lab Environment}\label{lab-environment}}

\hypertarget{lab-work}{%
\section{Work Expectations}\label{lab-work}}

\hypertarget{lab-timeoff}{%
\subsection{Time Off}\label{lab-timeoff}}

\hypertarget{lab-communication}{%
\subsection{Communication Expectations}\label{lab-communication}}

\hypertarget{lab-mentorship}{%
\section{Mentorship}\label{lab-mentorship}}

\hypertarget{conflicts}{%
\section{Resolving Conflicts}\label{conflicts}}

\hypertarget{research-policies}{%
\chapter{Research Policies}\label{research-policies}}

\hypertarget{academic-integrity}{%
\section{Academic Integrity}\label{academic-integrity}}

\hypertarget{authorship}{%
\section{Authorship}\label{authorship}}

\hypertarget{open-science}{%
\section{Open Science}\label{open-science}}

\hypertarget{sharing-experiment-code-and-data}{%
\subsection{Sharing experiment code and data}\label{sharing-experiment-code-and-data}}

\hypertarget{preprints-and-conference-presentations}{%
\subsection{Preprints and conference presentations}\label{preprints-and-conference-presentations}}

\hypertarget{lab-resources}{%
\chapter{Lab Resources}\label{lab-resources}}

\hypertarget{required-tools}{%
\section{Required Tools}\label{required-tools}}

\hypertarget{recommended}{%
\section{Recommended}\label{recommended}}

\hypertarget{optional}{%
\section{Optional}\label{optional}}

\hypertarget{getting-started}{%
\chapter{Yokoyama Lab Onboarding}\label{getting-started}}

\hypertarget{before-you-join}{%
\section{Before you Join}\label{before-you-join}}

\hypertarget{first-week-tasks}{%
\section{First Week Tasks}\label{first-week-tasks}}

\hypertarget{first-six-months}{%
\section{First Six Months}\label{first-six-months}}

\hypertarget{miscellaneous}{%
\chapter{Miscellaneous}\label{miscellaneous}}

\hypertarget{collaborators}{%
\section{Collaborators}\label{collaborators}}

\hypertarget{extra-resources}{%
\section{Additional resources}\label{extra-resources}}

  \bibliography{book.bib,packages.bib}

\end{document}
